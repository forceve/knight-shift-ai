% !TeX program = xelatex
\documentclass[12pt]{article}
\usepackage[a4paper,margin=1in]{geometry}
\usepackage{hyperref}
\usepackage{enumitem}
\usepackage{fontspec}
\usepackage{xeCJK}
\setmainfont{TeX Gyre Termes}
\setCJKmainfont{Noto Serif CJK SC}

\title{\textbf{Night Shift AI Studio — Optimized Full Talk Script (Bilingual CN/EN)}}
\author{Grey}
\date{}

\begin{document}
\maketitle

\section*{Usage Note(使用方式)}
建议你现场 \textbf{以英文为主},每个 Scene 用一句中文做 ``punchline''(钉住主张),这样既不跳戏也更容易控时。

\section*{Scene0 — Title / Claim (0:00--0:20)}
EN: Hi everybody, good morning. My project is \textbf{Night Shift}. \\
EN: The key idea is simple: \textbf{Difficulty = computation budget}. \\
EN: With more budget, the engine sees further, becomes more stable, and avoids traps.

\medskip
CN(punchline): 我今天讲的不是“更强棋力”,而是\textbf{可控难度系统}。

\medskip
EN (optional short pun): I call it \textit{Night Shift} because it’s built with \textit{knights} --- and I built it mostly at night.

\section*{Scene1 — Freeze: Same board, different choice (0:20--1:20)}
EN: Let’s start with a quick question. \\
EN: Here is one position --- \textbf{same board, same side to move}. \\
EN: The opponent’s queen is \textbf{hanging}. \\
EN: \textbf{Would you take the queen?}

\medskip
(\textit{Pause 2 seconds. Look at the audience.})

\medskip
EN: Now watch what two levels do with different budgets. \\
EN: Left: the low-budget level takes immediately --- looks great. \\
EN: But a few moves later, it falls into a tactical trap. \\
EN: Right: the higher-budget level plays differently --- it sees the trap first.

\medskip
CN(punchline): 同一局面,不同选择——\textbf{因为预算不同}。

\section*{Scene2 — Difficulty Dial: four levels as budget presets (1:20--2:05)}
EN: I implemented \textbf{four difficulty levels} by increasing computation budget and heuristic precision. \\
EN: Levels are not arbitrary names --- they are \textbf{budget presets}.

\begin{itemize}[leftmargin=1.2em]
\item EN: \textbf{L1 Greedy (1-ply)}: very fast, very short-sighted.
\item EN: \textbf{L2 Minimax + Alpha-Beta (depth 3)}: material-only eval + small random tie-break.
\item EN: \textbf{L3 Practical engine}: alpha-beta + \textbf{quiescence} + \textbf{transposition table}, \textbf{0.6s/move}.
\item EN: \textbf{Ultimate}: same as L3 but stronger eval (\textbf{PeSTO}) + \textbf{1.2s/move}.
\end{itemize}

CN(punchline): 四档不是拍脑袋,是\textbf{预算预设}。

\section*{Scene3 — Knobs: why budget changes behavior (2:05--3:05)}
EN: Difficulty is not one knob. It’s four knobs:
\begin{enumerate}[leftmargin=1.4em]
\item EN: \textbf{Horizon} (depth / iterative deepening)
\item EN: \textbf{Efficiency} (alpha-beta, ordering, TT)
\item EN: \textbf{Eval richness} (material $\rightarrow$ positional $\rightarrow$ PeSTO)
\item EN: \textbf{Randomness} (controlled noise)
\end{enumerate}

CN(punchline): \textbf{Levels = presets of knobs},所以难度可控、可解释、可复现。

\section*{Scene4 — Ladder: the behavior staircase (3:05--3:45)}
EN: Think of the levels as a ladder:
\begin{itemize}[leftmargin=1.2em]
\item EN: L1 is fast and impulsive.
\item EN: L2 sees simple tactics, misses deeper threats.
\item EN: L3 is practical and stable.
\item EN: Ultimate has stronger evaluation and more time, so it converts advantages more reliably.
\end{itemize}

CN(punchline): 这不是强弱列表,是\textbf{行为阶梯}:越往上越稳。

\section*{Scene5 — Search X-Ray: internal mechanism (3:45--5:05)}
EN: What changes inside?
\begin{itemize}[leftmargin=1.2em]
\item EN: \textbf{PV} shows the best line.
\item EN: \textbf{Alpha-beta} prunes irrelevant branches.
\item EN: \textbf{Iterative deepening} stabilizes the best move under time limits.
\item EN: \textbf{Quiescence} reduces horizon effect in tactical positions.
\end{itemize}

CN(punchline): 预算不是“等更久”,而是让它\textbf{看得更远、更稳}。

\section*{Scene6 — Evaluation Harness: fair, reproducible, scalable (5:05--6:05)}
EN: To validate the ladder, I built an evaluation harness.
\begin{itemize}[leftmargin=1.2em]
\item EN: Same scoring: win 1 / draw 0.5 / loss 0.
\item EN: Colors reversed for fairness.
\item EN: Max 500 plies.
\end{itemize}

EN: Three protocols: M2M standard (100 games per pairing), H2M (10 participants), and time-scaled budgets. \\
CN(punchline): 这一页是我\textbf{结果可信}的担保。

\section*{Scene7 — Evidence Wall + TSB strip (6:05--8:20)}
EN: Evidence poster 1: round-robin scoreboard shows \textbf{separation}. \\
EN: Example highlights: \textbf{L2 vs L1 = 0.85}, \textbf{L3 vs L2 = 0.75}. \\
EN: Evidence poster 2: cost--strength curve shows \textbf{diminishing returns}. \\
EN: TSB strip shows the trend is stable across different budgets.

CN(punchline): 结论很稳:\textbf{预算决定难度,而且收益递减}。

\section*{Scene8 — Future Work (8:20--9:00)}
EN: Future work: performance optimization, dynamic difficulty adjustment, and generalization to other board games. \\
CN(punchline): 路线清晰,而且仍然\textbf{可评测}。

\section*{Scene9 — Closing / Handoff (9:00--9:30)}
EN: To close: \textbf{Difficulty = computation budget}. \\
EN: I delivered a multi-level engine, a reproducible evaluation pipeline, and evidence of separation and diminishing returns. \\
EN: Thank you --- now I can open the live demo. Q\&A?

\medskip
CN(收尾钉一句): 谢谢大家,我现在打开 /play 现场演示。

\end{document}
